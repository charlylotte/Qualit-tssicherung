\documentclass[a4paper, 10pt]{scrartcl}

\usepackage[utf8]{inputenc}
\usepackage[ngerman]{babel}
\usepackage[T1]{fontenc}
\usepackage{graphicx}
\usepackage{float}
\usepackage{wrapfig}
\usepackage{fancyhdr}
\usepackage{listings}
\usepackage{color}
\usepackage{hyperref}

\definecolor{mygreen}{rgb}{0,0.6,0}
\definecolor{mygray}{rgb}{0.5,0.5,0.5}
\definecolor{mymauve}{rgb}{0.58,0,0.82}

\lstset{
basicstyle=\footnotesize, 
commentstyle=\color{mygreen},
frame=single,	
language=Java,
breaklines=true,
numbers=left,
numberstyle=\tiny\color{mygray},
showtabs=false,
tabsize=2
}

\title{Jira in der Qualitätssicherung}
\author{Charlotte Probst (Matrikel Nr. 55607)\\
 Simon Westhoff (Matrikel Nr. 56327)}
\date{03.04.2019}
\pagestyle{fancy}

\lhead{Jira in der Qualitätssicherung}
\rhead{Charlotte Probst, Simon Westhoff}
\lfoot{Semester 7}
\cfoot{\thepage}
\rfoot{Prof. Dr. rer. nat. Dirk Hoffmann}

\renewcommand{\headrulewidth}{0pt}
\renewcommand{\footrulewidth}{0pt}

\begin{document}

\maketitle

%\begin{figure}[H]
%	\centering
%	\includegraphics[width=0.5\textwidth]{Bilder/gaussian_blur.jpg}
%	\caption{Anwendung des Gauß-Filters}
%\end{figure}

\newpage

\tableofcontents

\newpage

\section{Einführung in Jira}
Jira ist eine Webanwendung die heutzutage in fast jedem agilen Projekt zum Einsatz kommt. JIRA ist eine Software zur Vorgangs- und Projektverfolgung, sie begleitet meist den gesamten Lebenszyklus eines Projekts. Sie wurde in Java programmiert und 2002 von Atlassian released. Atlassian ist ein Anbieter von Softwarelösungen für Softwareentwickler. Es hat sich im Laufe der Jahre mit mehreren namhaften Webanwendungen wie Bamboo, BitBucket und anderen Produkten für agiles Software Development zu einem unter Entwicklern sehr geschätzten Softwareunternehmen entwickelt.
\begin{itemize}
\item Jire begleitet gesamten Lebenszyklus eines Projektes oder Produktes.
\item Von der Phase der Ideensammlung über Konzeption und Umsetzung bis hin zu Fehlermanagement und Support kann Jira in jeder Phase der Wertschöpfungskette eines Unternehmens eingesetzt werden.
\item Jira bietet phasenübergreifende Konsistenz: Anforderungen, Aufgaben, Support-Tickets, Fehler etc. werden im selben System verwaltet und können dadurch miteinander in Beziehung gesetzt sowie gemeinsam durchsucht und ausgewertet werden.
\end{itemize}

\subsection{Kategorien zur Einordung unterschiedlicher Aufgaben (Vorgangstypen)}
In Jira werden verschieden Aufgabe mit Hilfe von Vorgangstypen kategorisiert. Es gibt 4 Kategorien:
\begin{enumerate}
\item \textbf{EPIC}\\
ist eine User-Story die sich über mehr als einen Sprint erstreckt.
\item \textbf{STORY}\\
besteht aus mehreren TASKs, ist eine USER-Story die sich in einem Sprint erledigen lässt ,man sollte sie in einem einfachen Satz beschreiben können.
\item \textbf{TASK}\\
entspricht einem einzelnen Aufgabenpaket.
\item \textbf{BUG}\\
ist ein Fehler, den es zu beheben gilt.
\end{enumerate}


\section{Einordnung im QS-Umfeld}

\section{Welche Funktionen bietet Jira?}

\section{Fazit}

\end{document}
